\subsection*{Introduction}

Une partie importante du travail de modélisation que nous avons effectué
s'est centrée sur les filtres passe-haut et passe-bas inclus dans notre circuit
et les alternatives existantes.

Les filtres auxquels nous nous sommes intéressés sont caractérisés
par ces propriétés:
\begin{itemize}
    \item la tension d'entrée est sinusoïdale;
    \item les filtres utilisent exclusivement des résistances,
        des capacités et des inductances;
    \item la tension de sortie est prise entre deux points arbitraires
        du circuit.
\end{itemize}
L'objet de notre étude a été le rapport des tensions d'entrée et sortie,
c'est-à-dire le quotient de leurs amplitudes ainsi que leur déphasage.


\subsubsection*{Plan du chapitre}
\begin{enumerate}
    \item Nous commencerons par la modélisation
        d'un filtre passe-haut en nous basant uniquement sur les lois
        de Kirchhoff et les équations caractéristiques des composants.
    \item Puis nous introduirons la notion d'impédance et son utilité
        dans le calcul des gains et déphasages.
    \item Ensuite, nous étudierons grâce à cette notion
        les différents filtres que nous avons découverts à travers
        la recherche bibliographique; nous en profiterons pour justifier
        le choix des filtres passe-haut et passe-bas dans notre circuit.
    \item Enfin, nous extrairons de ces exemples une manière de caractériser
        n'importe quel filtre par un rapport de fonctions
        polynômiales de la fréquence, et nous montrerons comment construire
        un filtre à partir de deux polynômes arbitraires donnés.
        (Pas sûr qu'on sait faire ça.)
\end{enumerate}
