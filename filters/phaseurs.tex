\section{Méthode des phaseurs}

Dans de nombreux domaines de la physique,
on retrouve des phénomènes périodiques.
Pour en simplifier leur étude,
on ramène les fonctions périodiques à une somme de sinusoïdales,
puis on étudie ces dernières avec la méthode des phaseurs.
\begin{itemize}
    \item La première étape consiste à prendre la transformée de Fourier,
        de la fonction périodique quelconque donnée.
        Toutefois, elle demande un développement d'analyse assez important.
        Nous n'étudierons ici que des signaux d'entrée sinusoïdaux purs
        (qui correspondent d'ailleurs à des sons purs).
    \item La deuxième étape consiste à représenter chacune des sinusoïdales
        sous une forme qui simplifie les calculs classiques à effectuer,
        c'est à dire la résolution d'équations différentielles,
        puis plus spécifiquement la résolution de circuits électriques.
        C'est celle que nous allons introduire ici.
\end{itemize}

Pour illustrer les phaseurs, nous allons utiliser l'exemple précédent
du filtre passe-bas,
mais cette méthode s'applique à bien d'autres phénomènes,
à commencer par les oscillateurs harmoniques,
comme nous le montrerons plus tard dans
une analyse simplifée de la propagation du son.
