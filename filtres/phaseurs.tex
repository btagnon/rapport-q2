\section{Méthode des phaseurs}

Dans de nombreux domaines de la physique,
on retrouve des phénomènes périodiques.
Une manière simplifier leur étude
est de ramèner les fonctions périodiques à une somme de sinusoïdales,
puis d'étudier ces dernières avec la méthode des phaseurs.
\begin{itemize}
    \item La première étape consiste à prendre la transformée de Fourier,
        de la fonction périodique quelconque donnée.
        Toutefois, elle demande un développement d'analyse assez important.
        Nous n'étudierons ici que des signaux d'entrée sinusoïdaux purs
        (qui correspondent d'ailleurs à des sons purs).
    \item La deuxième étape consiste à représenter chacune des sinusoïdales
        sous une forme qui simplifie les calculs classiques à effectuer,
        c'est à dire la résolution d'équations différentielles,
        puis plus spécifiquement la résolution de circuits électriques.
        C'est celle que nous allons introduire ici.
\end{itemize}

Pour illustrer les phaseurs, nous allons utiliser l'exemple précédent
du filtre passe-bas,
mais cette méthode s'applique à bien d'autres phénomènes,
à commencer par les oscillateurs harmoniques,
comme nous le montrerons plus tard dans
une analyse simplifée de la propagation du son.

\subsection{Définition de l'isomorphisme}
Commençons par étudier l'ensemble des fonctions du temps
\[
    f(t) = A\cos(\omega t + \phi)
\]
où $A$ est une constante réelle (avec éventuellement une dimension physique
\footnote{
    En réalité, le support complet des grandeurs physiques
    nécessite une extension de la définition d'espace vectoriel.
    En effet, seules des valeurs de mêmes unités sont additionnables,
    et la multiplication par un scalaire possédant des unités
    va changer les unités de la fonction.
    Dans la suite de ce raisonnement,
    nous procéderons comme si toutes les valeurs étaient sans unités.
}),
$\phi$ est une constante réelle en radians,
$t$ est une variable en secondes
et $\omega$ est un paramètre de l'ensemble, en Hertz
(c'est le même pour toutes les fonctions de l'ensemble).
Notons cet ensemble $\mathbb{S}_\omega$.

Il s'agit d'un espace vectoriel.
Nous ne détaillerons pas ici la preuve
(qui se trouve dans l'annexe~\ref{app:espace-vect-sinus}),
mais en pratique cela signifie qu'une combinaison linéaire
de sinusoïdales reste une sinusoïdales,
et que ces fonctions ont un certain nombre de propriétés
typiquement associées aux vecteurs.

Revenons à notre but de représenter des sinusoïdes
par des objets mathématiques plus simples.
Les exponentielles complexes allient un caractère sinusoïdal à
la simplicité d'une exponentielle.
Elles sont donc un bon candidat potentiel.
Nous savons que:
\[
    f(t) = A\cos(\omega t + \phi) = \mathfrak{Re}\{Ae^{j(\omega t + \phi)}\}
    = \mathfrak{Re}\{Ae^{j\phi}\cdot e^{j\omega t}\}
\]
où $j$ est l'unité imaginaire.
On remarque que $e^{j\omega t}$ ne dépend pas de la sinusoïde représentée.
Cela signifie que la constante complexe $Ae^{j\phi}$ contient
toute l'information qui caractérise la fonction $f$.

On peut même aller plus loin: définissons l'application $L$
qui à une fonction $f(t) = A\cos(\omega t + \phi)$
associe le complexe $Ae^{j\phi}$,
qu'on appellera le \emph{phaseur}.
On peut montrer (voir l'annexe~\ref{app:isomorphisme}) qu'il s'agit d'un
isomorphisme entre $S_\omega$ et $\mathbb{C}$.

Cela a deux conséquences importantes:
\begin{itemize}
    \item À toute fonction sinusoïdale correspond un et un seul phaseur
        (un isomorphisme est une bijection).
    \item On peut effectuer des sommes ou des multiplications par un scalaire
        indifféremment sur les fonctions ou sur les phaseurs correspondants.
\end{itemize}

On peut déjà voir les avantages de cette représentation.
En effet,
les opérations sur les complexes sont bien plus simples et mieux connues
que celles sur les fonctions.

Dans les sections suivantes nous montrerons
encore d'autres avantages.
