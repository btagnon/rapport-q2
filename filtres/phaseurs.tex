\section{Méthode des phaseurs}

Dans de nombreux domaines de la physique,
on retrouve des phénomènes périodiques.
Pour en simplifier leur étude,
on ramène les fonctions périodiques à une somme de sinusoïdales,
puis on étudie ces dernières avec la méthode des phaseurs.
\begin{itemize}
    \item La première étape consiste à prendre la transformée de Fourier,
        de la fonction périodique quelconque donnée.
        Toutefois, elle demande un développement d'analyse assez important.
        Nous n'étudierons ici que des signaux d'entrée sinusoïdaux purs
        (qui correspondent d'ailleurs à des sons purs).
    \item La deuxième étape consiste à représenter chacune des sinusoïdales
        sous une forme qui simplifie les calculs classiques à effectuer,
        c'est à dire la résolution d'équations différentielles,
        puis plus spécifiquement la résolution de circuits électriques.
        C'est celle que nous allons introduire ici.
\end{itemize}

Pour illustrer les phaseurs, nous allons utiliser l'exemple précédent
du filtre passe-bas,
mais cette méthode s'applique à bien d'autres phénomènes,
à commencer par les oscillateurs harmoniques,
comme nous le montrerons plus tard dans
une analyse simplifée de la propagation du son.

\subsection{Définition de l'isomorphisme}
Commençons par étudier l'ensemble des fonctions du temps
\[
    f(t) = A\cos(\omega t + \phi)
\]
où $A$ est une constante réelle (avec éventuellement une dimension physique
\footnote{
    En réalité, le support complet des grandeurs physiques
    nécessite une extension de la définition d'espace vectoriel.
    En effet, seules des valeurs de mêmes unités sont additionnables,
    et la multiplication par un scalaire possédant des unités
    va changer les unités de la fonction.
    Dans la suite de ce raisonnement,
    nous procéderons comme si toutes les valeurs étaient sans unités.
}),
$\phi$ est une constante réelle en radians,
$t$ est une variable en secondes
et $\omega$ est un paramètre de l'ensemble, en Hertz
(c'est le même pour toutes les fonctions de l'ensemble).

Il s'agit d'un espace vectoriel.
Nous ne détaillerons pas ici la preuve complète
(qui se trouve dans l'annexe \ref{}),
et nous nous limiterons à prouver que l'addition est interne.
Pour des fonctions sinusoïdales $f$ et $g$, on a:
\[
    \begin{array}{rcl}
        (f+g)(t) &=& A\cos(\omega t + \phi) + B\cos(\omega t + \psi) \\
                 &=& (A\cos\phi+B\cos\psi)\cos(\omega t)
                     - (A\sin\phi+B\sin\psi)\sin(\omega t)
    \end{array}
\]

Calculons l'amplitude résultante:
\[
    \begin{array}{rcl}
        C &=& \sqrt{(A\cos\phi+B\cos\psi)^2+(A\sin\phi+B\sin\psi)^2} \\
          &=& \sqrt{A^2+B^2+2AB\cos(\phi-\psi)}
    \end{array}
\]
qu'on peut également trouver grâce à la loi des cosinus.
On peut donc réécrire la somme:
\[
    (f+g)(t) = C\left[ \frac{A\cos\phi+B\cos\psi}{C}\, \cos(\omega t)
                     - \frac{A\sin\phi+B\sin\psi}{C}\, \sin(\omega t) \right]
\]

Enfin, puisque $\left( \frac{A\cos\phi+B\cos\psi}{C} \right)^2
+ \left( \frac{A\sin\phi+B\sin\psi}{C} \right)^2 = 1$,
il existe un réel $\theta$ tel que:
\[
    \left\{
    \begin{array}{rcl}
        \cos\theta &=& \frac{A\cos\phi+B\cos\psi}{C} \\
        \sin\theta &=& \frac{A\sin\phi+B\sin\psi}{C}
    \end{array}
    \right.
\]
et donc:
\[
    \begin{array}{rcl}
        (f+g)(t) &=& C ( \cos\theta\cos(\omega t)
                       - \sin\theta\sin(\omega t) ) \\
                 &=& C\cos(\omega t + \theta)
    \end{array}
\]
