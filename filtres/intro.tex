\subsection*{Introduction}

Une partie importante du travail de modélisation que nous avons effectué
s'est centrée sur les filtres de fréquences,
en commençant par les filtres passe-haut et passe-bas inclus dans notre circuit,
puis en élargissant vers d'autres types de filtres que nous avons découvert
dans nos recherches bibliographiques.

Les filtres auxquels nous nous sommes intéressés sont caractérisés
par ces propriétés communes:
\begin{itemize}
    \item ils utilisent exclusivement des résistances,
        des capacités et des inductances;
    \item la tension d'entrée est sinusoïdale;
    \item la tension de sortie est prise entre deux points arbitraires
        du circuit.
\end{itemize}

L'objet de notre étude a été la comparaison des tensions d'entrée et sortie,
c'est-à-dire le rapport de leurs amplitudes ainsi que leur déphasage,
en fonction de la fréquence du signal.


\subsubsection*{Plan du chapitre}
\begin{enumerate}
    \item Nous commencerons par la modélisation
        d'un filtre passe-bas par la méthode classique, en résolvant une
        équation différentielle de la tension.
    \item Puis nous introduirons la notation complexe des fonctions sinusoïdales
        et plus particulièrement la notion d'impédance et son utilité
        dans le calcul des gains et déphasages.
    \item Ensuite, nous étudierons grâce à cette notion
        les différents filtres que nous avons découverts à travers
        la recherche bibliographique;
        nous en profiterons pour discuter
        le choix des filtres dans notre circuit.
    \item Enfin, nous extrairons de ces exemples une manière de caractériser
        n'importe quel filtre par un rapport de fonctions
        polynômiales de la fréquence, et nous montrerons comment construire
        un filtre à partir de deux polynômes arbitraires donnés.
        (Pas sûr qu'on sait faire ça.)
\end{enumerate}
