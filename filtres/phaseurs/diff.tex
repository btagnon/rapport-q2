\subsection{Différentiation}

Nous allons maintenant déterminer l'opération qu'il faut effectuer
sur le phaseur d'une fonction pour obtenir le phaseur de sa dérivée temporelle.

Prenons comme d'habitude une fonction $f(t) = F\cos(\omega t + \phi)$
de phaseur $\overline{F} = Fe^{j\phi}$
et dérivons-la:
\begin{equation}
    \frac{d}{dt}f(t) = \omega (-A\sin(\omega t + \phi))
    = \omega A \cos(\omega t + \pi/2 + \phi)
\end{equation}

Le phaseur correspondant $\overline{F'}$ vaut:
\begin{equation}
    \overline{F'} = \omega A e^{j(\pi/2 + \phi)} = \omega A e^{j\pi/2} e^{j\phi}
    = j\omega A e^{j\phi} = j\omega \overline{F}
\end{equation}

On découvre donc que quand on dérive une fonction sinusoïdale,
son phaseur est simplement multiplié par $j\omega$.
Cela a comme conséquence immédiate de simplifier énormément
le calcul de solutions particulières sinusoïdales d'équations différentielles.

Reprenons l'équation différentielle décrivant le filtre passe-bas
\eqref{eq:diff-passe-bas}:
\[
    v_{in} = RC\frac{dv_C}{dt} + v_C
\]
on peut la réécrire sous forme de phaseurs:
\begin{equation}
    \overline{V_{in}} = RC(j\omega)\overline{V_C} + \overline{V_C}
\end{equation}
ce qui donne immédiatement la solution:
\begin{equation}
    \overline{V_C} = \frac{\overline{V_{in}}}{1 + j\omega RC}
\end{equation}

On peut ensuite extraire l'amplitude et le déphasage de $v_C$
en calculant respectivement le module et l'argument de $\overline{V_C}$:
\begin{equation}
    \left\{
        \begin{array}{ccl}
            V_C &=& \left\|\,\overline{V_C}\,\right\|
            = V_{in}\,/\,\|1+j\omega RC\|
            = V_{in}\,/\sqrt{1 + (\omega RC)^2} \\
            \phi &=& \arg\left(\overline{V_C}\right) = -\arg(1+j\omega RC)
            = - \arctan(\omega RC)
        \end{array}
    \right.
\end{equation}
ce qui correspond exactement aux résultats trouvés précédemment.
