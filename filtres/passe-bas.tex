\section{Méthode classique: filtre passe-bas}

Le filtre passe-bas consiste en une résistance $R$ et une capacité $C$
mises en série,
comme illustré dans la figure~\ref{circ:passe-bas}.
Le signal d'entrée est reçu aux extrémités, et la tension de sortie est prise
aux bornes de la capacité.

Définissons:
\begin{itemize}
    \item $v_{in}(t)$ la tension d'entrée;
    \item $v_R(t)$ la tension aux bornes de la résistance;
    \item $v_C(t)$ la tension aux bornes de la capacité;
    \item $i(t)$ le courant (de haut en bas).
\end{itemize}

À partir d'ici nous les considérerons implicitement
comme des fonctions du temps.
Nous supposons que le circuit est en régime sinusoïdal stable.

Les données dont nous disposons sont:
\begin{itemize}
    \item Loi des tensions de Kirchhoff: $v_R + v_C = v_{in}$
    \item Résistance: $v_R = Ri$
    \item Capacité: $i = C\,dv_C/dt$
\end{itemize}

Développons la première relation:
\begin{equation}
    v_{in} = v_R + v_C = Ri + v_C = RC\,\frac{dv_C}{dt} + v_C
\end{equation}
Ceci est une équation différentielle linéaire d'ordre 1 en $v_C$,
à coefficients constants, et non-homogène.
La solution de l'équation homogène correspondante est $v_C = e^{-t/RC}$.
Toutefois, il s'agit d'une exponentielle décroissante,
donc son effet va s'estomper à l'état stable.

Posons $v_{in}(t) = V_{in}\cos(\omega t)$.
Puisque le signal d'entrée est sinusoïdal, on peut trouver une solution
particulière de la forme
\begin{equation*}
    v_C(t) = V_C\cos(\omega t + \phi)
    = (V_C\cos\phi)\,\cos(\omega t) - (V_C\sin\phi)\,\sin(\omega t)
\end{equation*}
En substituant ces expressions dans l'équation,
puis en séparant les termes en $\cos(\omega t)$ et $\sin(\omega t)$,
on obtient:
\begin{equation}
    \left\{
        \begin{array}{ccrcr}
            V_{in} &=& V_C\cos\phi &-& \omega RC\ V_C\sin\phi \\
            0 &=& -\omega RC\ V_C\cos\phi &-& V_C\sin\phi
        \end{array}
    \right.
\end{equation}
ou encore:
\begin{equation}
    \left\{
        \begin{array}{ccl}
            V_C\cos\phi &=& V_{in}\,/(1+(\omega RC)^2) \\
            V_C\sin\phi &=& V_{in}\,(-\omega RC) / (1+(\omega RC)^2)
        \end{array}
    \right.
\end{equation}
et donc:
\begin{equation}
    \left\{
        \begin{array}{ccl}
            V_C &=& V_{in}\,/ \sqrt{1+(\omega RC)^2} \\
            \phi &=& \arctan(-\omega RC)
        \end{array}
    \right.
\end{equation}
où $V_C$ est l'\emph{amplitude} et $\phi$ le \emph{déphasage}
du signal de sortie.
Ces valeurs sont représentées dans la figure~\ref{graph:passe-bas},
et seront discutées plus tard.

Cette méthode a certains désavantages.
Outre la longueur des calculs qu'elle nécessite,
elle n'est pas facilement automatisable.
En effet, nous avons pris un cas très simple où l'équation différentielle
vient naturellement.
Mais dans d'autres cas,
d'une part la simple expression d'une
des lois de Kirchoff ne suffit pas,
et d'autre part il faudra parfois dériver les équations
obtenues pour faire apparaître la variable choisie.

Dès lors, une méthode plus générale et plus efficace est nécessaire.
